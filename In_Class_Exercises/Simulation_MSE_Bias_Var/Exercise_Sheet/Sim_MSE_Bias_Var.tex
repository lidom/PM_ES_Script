\documentclass[a4paper]{article}
\usepackage[]{graphicx}
\usepackage{amsmath}
\usepackage[]{color}
\usepackage{enumitem}
%% maxwidth is the original width if it is less than linewidth
%% otherwise use linewidth (to make sure the graphics do not exceed the margin)
\makeatletter
\def\maxwidth{ %
  \ifdim\Gin@nat@width>\linewidth
    \linewidth
  \else
    \Gin@nat@width
  \fi
} 
\makeatother

\definecolor{fgcolor}{rgb}{0, 0, 0}
\newcommand{\hlnum}[1]{\textcolor[rgb]{0.69,0.494,0}{#1}}%
\newcommand{\hlstr}[1]{\textcolor[rgb]{0.749,0.012,0.012}{#1}}%
\newcommand{\hlcom}[1]{\textcolor[rgb]{0.514,0.506,0.514}{\textit{#1}}}%
\newcommand{\hlopt}[1]{\textcolor[rgb]{0,0,0}{#1}}%
\newcommand{\hlstd}[1]{\textcolor[rgb]{0,0,0}{#1}}%
\newcommand{\hlkwa}[1]{\textcolor[rgb]{0,0,0}{\textbf{#1}}}%
\newcommand{\hlkwb}[1]{\textcolor[rgb]{0,0.341,0.682}{#1}}%
\newcommand{\hlkwc}[1]{\textcolor[rgb]{0,0,0}{\textbf{#1}}}%
\newcommand{\hlkwd}[1]{\textcolor[rgb]{0.004,0.004,0.506}{#1}}%
\let\hlipl\hlkwb

\usepackage{framed}
\makeatletter
\newenvironment{kframe}{%
 \def\at@end@of@kframe{}%
 \ifinner\ifhmode%
  \def\at@end@of@kframe{\end{minipage}}%
  \begin{minipage}{\columnwidth}%
 \fi\fi%
 \def\FrameCommand##1{\hskip\@totalleftmargin \hskip-\fboxsep
 \colorbox{shadecolor}{##1}\hskip-\fboxsep
     % There is no \\@totalrightmargin, so:
     \hskip-\linewidth \hskip-\@totalleftmargin \hskip\columnwidth}%
 \MakeFramed {\advance\hsize-\width
   \@totalleftmargin\z@ \linewidth\hsize
   \@setminipage}}%
 {\par\unskip\endMakeFramed%
 \at@end@of@kframe}
\makeatother

\definecolor{shadecolor}{rgb}{.97, .97, .97} 
\definecolor{messagecolor}{rgb}{0, 0, 0}
\definecolor{warningcolor}{rgb}{1, 0, 1}
\definecolor{errorcolor}{rgb}{1, 0, 0}
\newenvironment{knitrout}{}{} % an empty environment to be redefined in TeX

\usepackage{alltt}
\usepackage{booktabs} 
\usepackage{color}
\usepackage{hyperref}
\definecolor{britishracinggreen}{rgb}{0.0, 0.26, 0.15}
\definecolor{ao}{rgb}{0.0, 0.0, 1.0}
\definecolor{frenchblue}{rgb}{0.0, 0.45, 0.73}
\definecolor{lapislazuli}{rgb}{0.15, 0.38, 0.61}
\hypersetup{urlcolor=lapislazuli, colorlinks=TRUE}
\usepackage[ngerman]{babel}
\usepackage[T1]{fontenc}        
\usepackage{ucs}
\usepackage[utf8x]{inputenc} 
\usepackage{marginnote}
%=== bibliography package ===
\usepackage{natbib}
\bibliographystyle{harvard}
%citet and citep
%=============================

\newcommand{\gqq}[1]{\glqq{}{#1}\grqq{}}

%===================================================  
\sloppy
\parindent0pt
\parskip1.5ex plus.5ex minus.5ex
\addtolength{\textheight}{2cm} \addtolength{\topmargin}{-1cm} \addtolength{\textwidth}{2cm}
\addtolength{\oddsidemargin}{-1cm}
\IfFileExists{upquote.sty}{\usepackage{upquote}}{}
\begin{document}



Universität Bonn \hfill   \\%\today
Research Module Econometrics \& Statistics \hfill JProf.~Dominik Liebl
%=====================================================
%\vspace {0.5cm}
\rule{\textwidth}{.2pt}
\begin{center} 
\textbf{{\large MSE, Bias$^2$, and Variance of the ML-Estimator\\of the Standard Deviation}}\\[1ex]
\end{center}
% \begin{center}
% \end{center}
\rule{\textwidth}{.2pt}
%====================================================


{\sc Setup: }
\begin{itemize}[leftmargin=0.3cm]
\item Assumptions: iid random sample $(X_1,\dots,X_n)$ with $X_i\sim N(\mu,\sigma^2)$ for all $i=1,\dots,n$.
\item Fix values for the mean $\mu$, the variance $\sigma^2$, and the sample size $n$. \\
For instance: $\mu=3$, $\sigma=1.5$, and $n=\{2,4,6,\dots,30\}$.
\end{itemize}

{\sc Monte-Carlo Algorithm:}
% Declare a container-variable \texttt{decisions <- integer(length = B)}, where \texttt{B} denotes the number of Monte-Carlo simulations. Choose a large number such as \texttt{B}$=10,000$.
%In the $\texttt{b}$th simulation run, with $\texttt{b}=1,\dots,\texttt{B}$, do the following:
\begin{enumerate}
\item Simulate a realization from the iid random sample $(X_1,\dots,X_n)$.
\item Compute $s_n=\sqrt{\frac{1}{n}\sum_{i=1}^n(X_i-\bar{X}_n)^2}$, where $\bar{X}_n=n^{-1}\sum_{i=1}^nX_i$
\end{enumerate}

Repeat Steps 1-2 a large number of times, e.g., $R=10,000$ times and save all estimates $s_{n,1},\dots,s_{n,R}$. Then approximate the Mean Squared Error (MSE), the squared bias (Bias$^2$), and the variance (Var) by
\begin{align*}
\operatorname{MSE}(s_n)   &\approx \frac{1}{R}\sum_{r=1}^R (s_{n,r}-\sigma)^2\\
\operatorname{Bias}^2(s_n)&\approx \left(\left(\frac{1}{R}\sum_{r=1}^R s_{n,r}\right)-\sigma\right)^2\\
\operatorname{Var}(s_n)&\approx\frac{1}{R}\sum_{r=1}^R \left(s_{n,r}-\left(\frac{1}{R}\sum_{r=1}^R s_{n,r}\right)\right)^2
\end{align*}
where $E(s_{n,r})\approx R^{-1}\sum_{r=1}^Rs_{n,r}$. (The \textit{Law of Large Numbers} implies that the approximations become arbitrarily precise for $R\to\infty$.)

\bigskip

{\sc Presentation of the Simulation-Results:}
\begin{itemize}[leftmargin=0.3cm]
\item Plot your results (y-axis: MSE, Bias$^2$, and Var; x-axis: $n$)
\item Add the corresponding results for the unbiased estimator $\tilde{s}_{n}=\sqrt{\frac{1}{n-1}\sum_{i=1}^n(X_i-\bar{X}_n)^2}$.
\end{itemize}
\end{document}
