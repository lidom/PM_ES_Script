\documentclass[a4paper]{article}
\usepackage[]{graphicx}
\usepackage[]{color}
\usepackage{enumitem}
%% maxwidth is the original width if it is less than linewidth
%% otherwise use linewidth (to make sure the graphics do not exceed the margin)
\makeatletter
\def\maxwidth{ %
  \ifdim\Gin@nat@width>\linewidth
    \linewidth
  \else
    \Gin@nat@width
  \fi
} 
\makeatother

\definecolor{fgcolor}{rgb}{0, 0, 0}
\newcommand{\hlnum}[1]{\textcolor[rgb]{0.69,0.494,0}{#1}}%
\newcommand{\hlstr}[1]{\textcolor[rgb]{0.749,0.012,0.012}{#1}}%
\newcommand{\hlcom}[1]{\textcolor[rgb]{0.514,0.506,0.514}{\textit{#1}}}%
\newcommand{\hlopt}[1]{\textcolor[rgb]{0,0,0}{#1}}%
\newcommand{\hlstd}[1]{\textcolor[rgb]{0,0,0}{#1}}%
\newcommand{\hlkwa}[1]{\textcolor[rgb]{0,0,0}{\textbf{#1}}}%
\newcommand{\hlkwb}[1]{\textcolor[rgb]{0,0.341,0.682}{#1}}%
\newcommand{\hlkwc}[1]{\textcolor[rgb]{0,0,0}{\textbf{#1}}}%
\newcommand{\hlkwd}[1]{\textcolor[rgb]{0.004,0.004,0.506}{#1}}%
\let\hlipl\hlkwb

\usepackage{framed}
\makeatletter
\newenvironment{kframe}{%
 \def\at@end@of@kframe{}%
 \ifinner\ifhmode%
  \def\at@end@of@kframe{\end{minipage}}%
  \begin{minipage}{\columnwidth}%
 \fi\fi%
 \def\FrameCommand##1{\hskip\@totalleftmargin \hskip-\fboxsep
 \colorbox{shadecolor}{##1}\hskip-\fboxsep
     % There is no \\@totalrightmargin, so:
     \hskip-\linewidth \hskip-\@totalleftmargin \hskip\columnwidth}%
 \MakeFramed {\advance\hsize-\width
   \@totalleftmargin\z@ \linewidth\hsize
   \@setminipage}}%
 {\par\unskip\endMakeFramed%
 \at@end@of@kframe}
\makeatother

\definecolor{shadecolor}{rgb}{.97, .97, .97} 
\definecolor{messagecolor}{rgb}{0, 0, 0}
\definecolor{warningcolor}{rgb}{1, 0, 1}
\definecolor{errorcolor}{rgb}{1, 0, 0}
\newenvironment{knitrout}{}{} % an empty environment to be redefined in TeX

\usepackage{alltt}
\usepackage{booktabs} 
\usepackage{color}
\usepackage{hyperref}
\definecolor{britishracinggreen}{rgb}{0.0, 0.26, 0.15}
\definecolor{ao}{rgb}{0.0, 0.0, 1.0}
\definecolor{frenchblue}{rgb}{0.0, 0.45, 0.73}
\definecolor{lapislazuli}{rgb}{0.15, 0.38, 0.61}
\hypersetup{urlcolor=lapislazuli, colorlinks=TRUE}
\usepackage[ngerman]{babel}
\usepackage[T1]{fontenc}        
\usepackage{ucs}
\usepackage[utf8x]{inputenc} 
\usepackage{marginnote}
%=== bibliography package ===
\usepackage{natbib}
\bibliographystyle{harvard}
%citet and citep
%=============================

\newcommand{\gqq}[1]{\glqq{}{#1}\grqq{}}

%===================================================  
\sloppy
\parindent0pt
\parskip1.5ex plus.5ex minus.5ex
\addtolength{\textheight}{2cm} \addtolength{\topmargin}{-1cm} \addtolength{\textwidth}{2cm}
\addtolength{\oddsidemargin}{-1cm}
\IfFileExists{upquote.sty}{\usepackage{upquote}}{}
\begin{document}



Universität Bonn \hfill   \\%\today
Research Module Econometrics \& Statistics \hfill JProf.~Dominik Liebl
%=====================================================
%\vspace {0.5cm}
\rule{\textwidth}{.2pt}
\begin{center} 
\textbf{{\large Powerfunction Simulation for the z-Test}}\\[1ex]
\end{center}
% \begin{center}
% \end{center}
\rule{\textwidth}{.2pt}
%====================================================


{\sc Setup: }
\begin{itemize}[leftmargin=0.3cm]
\item Assumptions: Gaussian iid random sample $(X_1,\dots,X_n)$, where $X_i\sim N(\mu,\sigma^2)$ for all $i=1,\dots,n$.
\item Test H$_0$: $\mu=\mu_0$ versus  H$_1$: $\mu\neq\mu_0$.
\item Fix values for the significance level $\alpha$, the mean $\mu$, the variance $\sigma^2$, and the sample size $n$. \\
For instance: $\alpha=0.05$, $\mu\in\{-2,-1.5,\dots,-0.5,0,0.5,\dots,1.5,,2\}$, $\sigma=1$, $\mu_0=0$, and $n=\{10,15,20\}$.
\end{itemize}

{\sc Monte-Carlo Algorithm:}
% Declare a container-variable \texttt{decisions <- integer(length = B)}, where \texttt{B} denotes the number of Monte-Carlo simulations. Choose a large number such as \texttt{B}$=10,000$.
%In the $\texttt{b}$th simulation run, with $\texttt{b}=1,\dots,\texttt{B}$, do the following:
\begin{enumerate}
\item Simulate a realization from the iid random sample $(X_1,\dots,X_n)$.
\item Compute $Z_{obs}=\frac{\sqrt{n}(\bar{X}-\mu_0)}{\sigma}$, where $\bar{X}=n^{-1}\sum_{i=1}^nX_i$
\item Save test-decision: 
\begin{center}
\begin{tabular}{llc}
Criterion                   & Decision             & Save Value\\ 
\toprule    
$|Z_{obs}|>q_{1-\alpha/2}$ & Rejection of $H_0$   & $D=1$\\
$|Z_{obs}|<q_{1-\alpha/2}$ & No rejection of $H_0$& $D=0$\\   
\bottomrule
\end{tabular}  
\end{center}  
\end{enumerate}

Repeat Steps 1-3 a large number of times, e.g., $R=10,000$ times and save all binary test-decisions $D_1,\dots,D_R$. Then 
$$
\beta_{n,\alpha}(\mu)\approx \bar{D}_R:=\frac{1}{R}\sum_{r=1}^R D_r,
$$
where the approximation error $|\beta_{n,\alpha}(\mu)-\bar{D}_R|$ can be made arbitrarily small as $R\to\infty$.

\bigskip

{\sc Theoretical Justification:}
\begin{itemize}[leftmargin=0.3cm]
\item $D:=1_{(|Z_{obs}|>q_{1-\alpha/2})}\in\{0,1\}$ is a Bernoulli random variable with  probabilities $P\left(D=1\right)=p$ \quad and \quad $P\left(D=0\right)=1-p$. 
\item Note that $p$ depends on $\alpha$, $n$, $\sigma^2$, $\mu$, and $\mu_0$.%, that is, $p\equiv p_{\alpha, n, \sigma^2, \mu, \mu_0}$. 
\item By the Weak Law of Large Numbers\footnote{See, for instance, \url{www.statlect.com/asymptotic-theory/law-of-large-numbers}} we have that 
\begin{center}
  $\displaystyle\bar{D}_R\to_{P}E(D)=p$\quad as\quad $\displaystyle R\to\infty,$
\end{center}
where $p=\beta_{n,\alpha}(\mu)$.
\item That is, for instance under H$_0:\mu=\mu_0$, we have that $p=\alpha$. 
\end{itemize}
\end{document}
